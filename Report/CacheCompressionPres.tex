%%%%%%%%%%%%%%%%%%%%%%%%%%%%%%%%%%%%%%%%%%%%%%%%%%%%%%%%%%%%%%%
%
% Welcome to Overleaf --- just edit your LaTeX on the left,
% and we'll compile it for you on the right. If you open the
% 'Share' menu, you can invite other users to edit at the same
% time. See www.overleaf.com/learn for more info. Enjoy!
%
%%%%%%%%%%%%%%%%%%%%%%%%%%%%%%%%%%%%%%%%%%%%%%%%%%%%%%%%%%%%%%%
\documentclass{beamer}
\usepackage{tikz}
\usetheme{Copenhagen}
\usecolortheme{default}

\title[XOR Cache Compression] %optional
{XOR Cache Compression}
\subtitle{Utilizing Compression Algorithms to Increase Cache Efficiency}
\author[Aliji, Antony, Craveiro, Sristy] % (optional)
{I.~Aliji \and C.~Antony \and L.~Craveiro \and Z.N.~Sristy}

\institute[VFU] 
{
  %
  College of Science and Engineering\\
  University of Minnesota - Twin Cities

}

\date[VLC 2021] % (optional)
{CSCI 5204, Fall 2025}

% Use a simple TikZ graphic to show where the logo is positioned
% \logo{\begin{tikzpicture}
% \filldraw[color=red!50, fill=red!25, very thick](0,0) circle (0.5);
% \node[draw,color=white] at (0,0) {LOGO HERE};
% \end{tikzpicture}}

%End of title page configuration block
%------------------------------------------------------------
%The next block of commands puts the table of contents at the 
%beginning of each section and highlights the current section:

\AtBeginSection[]
{
  \begin{frame}
    \frametitle{Table of Contents}
    \tableofcontents[currentsection]
  \end{frame}
}
%------------------------------------------------------------
\begin{document}
\frame{\titlepage}
%---------------------------------------------------------
%Highlighting text
\begin{frame}
\frametitle{Introduction}

Compression techniques are utilized in many facets of computing, increasing the efficiency of data transfers. Utilizing some of these methods on a processor's cache is a great use case. The goal of our research is to implement different compression algorithms on the cache. 





\end{frame}
\begin{frame}{Motivation - Cache Compression}
Cache Compression Goal: condense larger chunks of data into smaller blocks without losing any information

\medskip

Why cache compression?
\begin{itemize}
    \item Benefits of increasing cache size without latency concerns or other losses that can occur from larger cache sizes
\end{itemize}
    
\end{frame}

\begin{frame}{Motivation - Granularity}
Compression methods vary by compaction scheme

\medskip

One factor we chose to focus on:
\textbf{Granularity}
\begin{itemize}
    \item Granularity: whether deduplication occurs between values in a line or between the lines themselves.
    \item Intra-line reduces size of data being stored by exploiting internal line patterns
    \item Inter-line leverages similarity across multiple cache lines
    \item Some methods make use of a hybrid approach that uese both intra and inter line-compression
\end{itemize}

\medskip

\end{frame}

\begin{frame}{Motivation - Scholarship}

Previous methods of cache compression: 

\medskip

\begin{itemize}
    \item Base-Delta Immediate (2012): Leverage small differences between values stored in a cache line. Represent cache lines in more compact form; use base values and an array of differences between values. (intra-line)
    \item Base and Compressed Difference Deduplication (2021): Makes use of partial matches across blocks; high bits tend to have common values acorss blocks even if they are different within blocks. Series of compression and deduplication of partially matching blocks. (inter-line)
\end{itemize}

\medskip



\end{frame}

\begin{frame}{Motivation - XOR Cache}
    \textbf{Goal}: A feasible and simple method of cache compression that is still able to significantly improve cache performance

\medskip

Making use of both intra and inter-line redundancy, we chose to implement the \textbf{XOR Cache} (2025)
\end{frame}

\begin{frame}[allowframebreaks]{References} 
    \nocite{*}
    \bibliographystyle{unsrt}
    \bibliography{reference}
\end{frame}
\end{document}
